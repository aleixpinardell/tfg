\section*{Glosario}
\label{sec:glosario}

\paragraph{Alargamiento alar.} Estimación de la proporción existente entre la longitud y la anchura media de un ala, que se calcula como el cociente entre el cuadrado de la envergadura alar y la superficie alar.

\paragraph{Alcance.} Distancia recorrida por una aeronave para completar una misión.

\paragraph{Ángulo de flecha.} Ángulo formado entre ele ala y el eje lateral de una aeronave. Se considera positivo si la raíz del ala se encuentra más cerca del morro que las puntas del ala.

\paragraph{Corte por hilo caliente.} Técnica de construcción que permite la obtención de piezas a partir del corte de espumas como el poliestireno expandido o extruido.

\paragraph{Diedro.} Ángulo que forma cada una de las semialas de un ala con la horizontal. Se considera positivo si las puntas de ala se encuentran más elevadas que la raíz.

\paragraph{Efecto suelo.} Fenómeno aerodinámico que sucede cuando un cuerpo, con una diferencia de presiones entre la zona que hay por encima de él y la que hay por debajo, está muy cerca de una superficie subyacente, lo que provoca unas alteraciones en el flujo de aire que pueden aprovecharse en diversos campos.

\paragraph{Efecto suelo dominado por la cuerda.} Efecto suelo apreciable en una región que se extiende aproximadamente una cuerda alar por encima de la superficie subyacente y que se caracteriza por ocasionar un aumento de la sustentación y de la resistencia.

\paragraph{Efecto suelo dominado por la envergadura.} Efecto suelo apreciable en una región que se extiende aproximadamente una envergadura alar por encima de la superficie subyacente y que se caracteriza por ocasionar un aumento de la sustentación y una disminución de la resistencia.

\paragraph{Eficiencia aerodinámica.} Relación entre la sustentación y la resistencia generada por un cuerpo.

\paragraph{Ekranoplano.} Vehículo parecido a un avión que casi nunca abandona la zona de influencia del efecto suelo y que ha sido diseñado para aprovechar esta circunstancia.

\paragraph{Ekranoplano de tipo Lippisch.} Ekranoplano que dispone de un ala delta invertida, un alargamiento alar moderado, un diedro negativo y una cola de dimensiones relativamente reducidas situada en posición elevada, fuera de la zona de influencia del efecto suelo.

\paragraph{Envergadura alar.} Distancia existente entre las dos puntas de un ala.

\paragraph{PAR (\emph{Power Augmented Regime}).} Técnica que mejora las prestaciones de un vehículo durante el despegue mediante el direccionamiento de los gases de escape del motor hacia la parte inferior del vehículo.

\paragraph{Salto dinámico.} Abandono temporal de la zona de influencia del efecto suelo, habitualmente con el fin de superar un obstáculo en condiciones de seguridad.

\paragraph{Separación vertical (\emph{ground clearance}).} Distancia existente entre la parte inferior del fuselaje de un ekranoplano y la superficie subyacente.

\paragraph{Torbellino de punta de ala.} Patrón circular de aire creado por un ala al generar sustentación que ocasiona un incremento de la resistencia.

\paragraph{Vehículo WIG.} Ekranoplano.


