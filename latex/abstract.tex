Desde su concepción a mediados del siglo pasado, los vehículos que hacen uso del efecto suelo de manera continuada durante el vuelo han evolucionado lentamente sin llegar nunca a popularizarse. Los sucesivos modelos introducidos se han centrado en diseñar un vehículo que, volando a poca distancia de la superficie del mar, sea capaz de operar en condiciones de seguridad a una velocidad superior y con un consumo inferior al de las embarcaciones navales convencionales. Si bien este vehículo no puede competir con los aviones comerciales, el hecho de que pueda llegar a batir a los navíos en dos de los aspectos más relevantes del transporte —el tiempo y el dinero—, le confiere un prometer potencial de revolucionar el transporte marítimo intercontinental.

En este trabajo se presta especial atención a la configuración aerodinámica presentada por Alexander Lippisch a principios de los años 60, la cual se ha convertido en líder indiscutible dentro de los vehículos que vuelan continuadamente dentro la zona de influencia del efecto suelo. Con el objetivo de determinar experimentalmente las prestaciones de este tipo de vehículo, se propone el diseño y construcción de un modelo a escala que será sometido a diversas pruebas en las que se pretende cuantificar el beneficio que supone el hecho de volar cerca de una superficie sólida y estudiar, al menos de manera cualitativa, las prestaciones en estabilidad para distintas configuraciones.