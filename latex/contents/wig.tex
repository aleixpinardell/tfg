En esta sección se definirá el efecto suelo, explicando sus fundamentos físicos, para luego exponer las diferentes aplicaciones que tiene en el ámbito de la aeronáutica, destacando una de ellas por encima del resto: los ekranoplanos o vehículos WIG (\emph{wing in ground effect craft}). Tras un breve repaso de las principales ventajas y limitaciones asociados a este tipo de aeronave, se expondrán las distintas configuraciones existentes en base a criterios aerodinámicos, prestando especial atención a la que seguramente sea la más popular en este ámbito: el ekranoplano de tipo Lippisch.


\subsection{El efecto suelo}
\label{sec:wig:ge}

%% características de la superficie a la que se acerca.
El efecto suelo es un fenómeno aerodinámico experimentado por una superficie sustentadora al acercarse a otra superficie, generalmente el suelo o el mar. Habitualmente conlleva un incremento de la sustentación y una reducción de la resistencia inducida.

Históricamente, antes de que se conocieran los fundamentos físicos del efecto suelo, se explicaba como una especie de “cojín de aire” generado entre la aeronave y la pista, y a menudo se veía como un fenómeno negativo, pues dificultaba el aterrizaje. Sin embargo, a medida que se han realizado avances en el campo de la aerodinámica, se ha identificado el efecto suelo como un fenómeno potencialmente beneficioso, ya que aumenta la eficiencia aerodinámica, reduciendo el consumo de combustible.


\subsubsection{Principio físico}
\label{sec:wig:ge:principle}

La generación de sustentación por parte de un ala se basa en la creación de una diferencia de presiones entre la parte superior e inferior de la misma. Cuando la presión del aire situado bajo la superficie alar es mayor, se crea un flujo que tiende a dirigirse hacia la parte superior, donde la presión es menor, generando la sustentación. Sin embargo, cerca de los extremos del ala, el flujo “detecta” que no existe ningún obstáculo más allá de la envergadura alar y circula hasta la parte superior del ala por la zona exterior, generando un torbellino, \rmfig{wingtipvortices}{jpg}{90}{bt}{Torbellinos de punta de ala generados por una aeronave a su paso por una nube}. Estos torbellinos crean una resistencia añadida sobre la aeronave, denominada resistencia inducida.

Cuando la aeronave vuela cerca de una superficie, por ejemplo la pista de un aeródromo, los torbellinos de punta de ala no pueden desarrollarse libremente y, al encontrar un obstáculo, se “achatan”, alejándose del eje longitudinal de la aeronave, \rmfig{groundeffect}{jpg}{90}{bt}{Ilustración esquemática de los torbellinos de punta de ala de un avión en vuelo (izquierda) y la distorsión sufrida por a causa del efecto suelo (derecha)}. Esto tiene dos efectos inmediatos: el primero de ellos, un incremento de la envergadura efectiva, lo cual favorece la generación de sustentación; y el segundo, una reducción de la resistencia inducida experimentada por la aeronave, que a su vez comporta que con el mismo empuje constante la aeronave sea capaz de alcanzar una velocidad mayor y por lo tanto favorece también la generación de sustentación. Todos los efectos mencionados contribuyen a mejorar la eficiencia aerodinámica de la aeronave.


\subsubsection{Tipología}
\label{sec:wig:ge:tipology}

Es habitual distinguir entre dos tipos de efecto suelo, según la proximidad de la aeronave a la superficie subyacente: el primero de ellos se dice que actúa en el ámbito de la envergadura, y de se denomina en inglés \emph{span-dominated ground effect}. Se trata básicamente del fenómeno explicado hasta el momento. Sin embargo, cuando la aeronave se aproxima todavía más al suelo, se dice que el efecto suelo actúa en el ámbito de la cuerda, y de denomina \emph{chord-dominated ground effect}. En este caso, según la forma de la aeronave y del ala, puede acumularse aire bajo el ala que, ante la dificultad de “escapar”, aumenta la presión, lo cual conlleva un aumento de la sustentación pero también un aumento de la resistencia. En este caso, el efecto no siempre lleva consigo un aumento de la eficiencia aerodinámica.
%una referencia o dibujo seria genial aquí para explicar estas difernecias

En cualquier caso, es difícil determinar la distancia entre la aeronave y la superficie subyacente para la cual se considera que se entra en zona de influencia del efecto suelo dominado por la cuerda. Es más, suele ser difícil incluso la determinación de la región en la que el efecto suelo dominado por la envergadura tiene efectos significativos sobre la aeronave, pues depende de varios factores como la geometría de la aeronave, velocidad, actitud e incluso de las condiciones atmosféricas. En ambos casos, la determinación de las regiones espaciales en las que el efecto suelo es relevante pasa por ciertas definiciones no exentas de arbitrariedad, resultando imprescindible la realización de pruebas experimentales sobre la aeronave.


\subsubsection{Aplicaciones}
\label{sec:wig:ge:applications}

Algunas aeronaves se benefician del efecto suelo durante el despegue, lo que les permite acortar la distancia de despegue. Dado que el efecto suelo se acentúa al acercarse la superficie sustentadora –el ala– al suelo, la magnitud del fenómeno será mayor en aeronaves con ala baja. En estos, será fundamental diseñar y certificar la aeronave teniendo en cuenta que durante el aterrizaje es muy probable que el efecto suelo provoque un aumento en la distancia de pista requerida.

Además de en aeronaves de ala fija, el efecto suelo también se ha aprovechado en aeronaves de ala móvil, principalmente helicópteros. En este caso, cuando la aeronave se encuentra cerca del suelo, puede mantenerse en vuelo estacionario requiriendo una menor potencia.

No obstante, estos vehículos hacen uso del efecto suelo en situaciones muy concretas y por cortos períodos de tiempo, y en absoluto se benefician de él durante la fase de crucero, que representa prácticamente la totalidad del tiempo de vuelo. De ahí el potencial de los vehículos WIG, que sí aprovechan el efecto suelo durante la mayor parte o la totalidad del vuelo.


\subsection{El concepto de ekranoplano}
\label{sec:wig:ekranoplane}

Según las \emph{Interim Guidelines for WIG Craft}, un vehículo WIG (wing in ground) es un vehículo que vuela haciendo uso del efecto suelo sobre el agua u otra superficie, sin estar en contacto con ella, y que se mantiene en el aire gracias a la sustentación generada principalmente por un ala.

El nombre histórico para este tipo de vehículo es el de ekranoplano, aunque en las últimas décadas se ha promocionado el nombre de “WIG craft” desde las entidades certificadoras. En este documento se usarán indistintamente ambas denominaciones.

Las primeras aeronaves que hicieron un uso intencionado del efecto suelo surgieron en la década de los años 30. Se trataban de aeronaves pensadas para volar a poca distancia del la superficie del mar y, en la mayoría de los casos, disponían de un ala media-baja de alargamiento reducido, una cola horizontal lo suficientemente alta para no verse afectada por el efecto suelo y uno o varios motores cuyos gases de escape se dirigían hacia la parte inferior del vehículo, energizando el flujo entre éste y la superficie subyacente.


\subsubsection{Características}
\label{sec:wig:ekranoplane:characteristics}

Aunque hasta ahora se ha remarcado en varias ocasiones la principal característica de estos vehículos, que es el hecho de disponer de una mayor eficiencia aerodinámica, la introducción del concepto de ekranoplano va más allá, pues define una nueva categoría de vehículos completamente nueva, al operar en condiciones muy distintas. El hecho de volar al nivel del mar tiene las siguiente implicaciones:
\begin{itemize}
\item Desaparece la necesidad de presurizar la cabina, así como los esfuerzos debidos a la diferencia de presiones entre el interior de la aeronave y el exterior.
\item Despaarece la necesidad de disponer de infraestructuras que permitan el despegue y aterrizaje del vehículo.
\item Aumenta la seguridad operativa, ya que en cualquier momento la aeronave puede amerizar de inmediato ante una emergencia.
\end{itemize}

Sin embargo, el ekranoplano también cuenta con algunas deficiencias, principalmente:
\begin{itemize}
\item Requiere de una superficie relativamente llana y libre de obstáculos sobre la cual volar, por lo que su uso se limita al transporte entre puntos conectados por agua. En el futuro se podría estudiar el uso de estos vehículos sobre tierra firme, aunque para ello sería necesario crear una infraestructura propia, renunciando a una de las principales ventajas del transporte aéreo.
\item Se elimina un grado de libertad, la altitud, pues todos los ekranoplanos deben volar a una altitud de vuelo similar, mientras que las aeronaves convencionales pueden operar a distintas altitudes de vuelo. A priori esto no supondría un problema, pero si en el futuro los ekranoplanos se llegan a popularizar, esta limitación podría tener consecuencias graves de congestión del “espacio marítimo”.
\end{itemize}


\subsubsection{Limitaciones}
\label{sec:wig:ekranoplane:limitations}

A continuación de exponen algunos de los factores limitantes que históricamente han impedido la popularización del ekranoplano.

\paragraph{Estabilidad y control.} El concepto de ekranoplano requiere llevar el sistema de estabilidad y control un paso más allá. En la aviación convencional, se debe asegurar que la aeronave tiende a recuperar su estado original ante una perturbación en cualquiera de sus tres ejes: longitudinal, lateral o direccional.

Sin embargo, en el caso de los ekranoplanos, aparece una nueva fuente de inestabilidad, asociada a una perturbación en altura o separación del suelo. Ante la presencia de una ola, la distancia entre la aeronave y la superficie subyacente se reduce, acentuándose el efecto suelo y aumentando la sustentación. Un diseño aceptable garantizará que el piloto pueda responder ante tal perturbación manteniendo el control de la aeronave. Un diseño óptimo deberá incluir además un sistema de estabilización automático que permita mantener una altura constante, siendo imperceptible tanto para la tripulación como para los pasajeros la existencia de crestas y valles en la superficie marítima.

Cabe destacar que un sistema como el mencionado no existía en la época en la que se empezó a experimentar con el concepto de ekranoplanos, ni siquiera en las décadas posteriores. Hoy en día, sin embargo, la mayoría de aviones convencionales incorporan avanzados sistemas aviónicos que facilitan la estabilidad y el control automático de los mismos.

\paragraph{Despegue.} Durante el despegue, los ekranoplanos requieren un empuje extra para vencer la resistencia generada por los elementos que se encuentran sumergidos o en contacto con el agua. Aunque esta limitación se puede solucionar incorporando un motor más potente, con ello se sacrifica buena parte de la eficiencia en crucero, pues durante esta fase se dispone de un motor sobredimensionado, y por lo tanto más pesado y con una resistencia parásita mayor.

Por ello, desde los primeros diseños experimentales, se ha buscado solucionar este problema mediante el desarrollo de la tecnología PAR –\emph{power augmented regime}–, que consiste fundamentalmente en dirigir el flujo saliente del motor hacia la parte inferior de la aeronave, creando una zona presurizada entre ésta y la superficie marítima, y consiguiendo con ello una mayor generación de sustentación y un despegue más corto. Sin embargo, esta solución limita la flexibilidad a la hora de ubicar del motor, e incrementa el riesgo de que éste absorba agua en exceso al deber estar situado en zonas bajas.

Otros ingenieros intentaron solucionar este problema utilizando plantas propulsoras distintas para el crucero y para el despegue, aunque resulta totalmente ineficiente cargar durante el crucero con el peso de una planta propulsora que está inoperativa.

\paragraph{Materiales.} Un vehículo que opera en las proximidades de la superficie marítima debe prestar especial atención a la aparición de posibles problemas de corrosión. Este hecho limita el uso de materiales en la fabricación de un ekranoplano a metales y aleaciones que sean altamente resistentes a la oxidación.

En las últimas décadas se ha introducido el uso de materiales compuestos en el ámbito aeronáutico con muy buenos resultados, al ser por lo general más ligeros que los materiales metálicos. Además, cuentan con una alta resistencia a ambientes corrosivos, lo cual aparentemente los haría idóneos para su uso en ekranoplanos. Sin embargo, a pesar de su elevada resistencia, no son buenos absorbiendo energía, por lo que su uso en partes del vehículo que puedan entrar en contacto con el agua durante el despegue o el aterrizaje es peligroso, y son precisamente éstas las partes que más expuestas se encuentran a la corrosión.


\subsubsection{Tipología}
\label{sec:wig:ekranoplane:tipology}

Existen una gran multitud de criterios en base a los cuales clasificar los distintos tipos de ekranoplanos. Una de las clasificaciones más relevantes desde el punto de vista ingenieril se realiza atendiendo a la configuración aerodinámica elegida. Según esta clasificación, destacan cuatro tipos de ekranoplanos:

\begin{itemize}
\item \textbf{Configuración tándem}. La estabilidad de la aeronave se consigue mediante el ajuste durante el vuelo de los ángulos de las diferentes superficies alares, tanto delanteras como posteriores. Su principal inconveniente es que solo es estable en un reducido rango de ángulos de ataque y separación hasta la superficie subyacente.
\item \textbf{Configuración de avión}. En este caso se emplea un ala baja a menudo de dimensiones relativamente grandes, y la cola se sitúa en una posición elevada con el objetivo de que no se vea afectada por el efecto suelo. El caso específico más destacado dentro de esta categoría, el ekranoplano de tipo Lippisch, mejora sustancialmente la estabilidad, permitiendo operar con seguridad en un rango de ángulos de ataque y distancia al suelo más amplio. El principal inconveniente de este grupo es el aumento de peso asociado al uso de una cola en T alejada del fuselaje.
\item \textbf{Configuración de ala volante}. Se eliminan o reducen sustancialmente la mayor parte de los elementos no sustentadores, reduciendo el peso en vacío de la aeronave de manera sustancial. Sin embargo, esta configuración cuenta con una estabilidad limitada a un cierto rango de ángulos de ataque y alturas.
\item \textbf{Configuración de ala compuesta}. En este caso se intenta aprovechar las mejores características de las configuraciones de avión y ala volante. Mediante un diseño cuidadoso del perfil alar del ala principal, se trata de reducir las dimensiones y el peso de la cola sin sacrificar la estabilidad del vehículo.
\end{itemize}


\subsection{El ekranoplano Lippisch}
\label{sec:wig:lippisch}

Algunas de las limitaciones expuestas en la \sectionref{wig:ekranoplane:limitations} se pudieron atajar con mayor o menor éxito durante las primeras décadas de existencia de los ekranoplanos. Por ejemplo, algunos ingenieros se decantaron por materiales inoxidables con propiedades mecánicas inferiores a otros materiales disponibles en el mercado pero inadecuados para esta aplicación por su escasa resistencia a la corrosión, y optaron por utilizar plantas propulsoras distintas durante el despegue y el resto de la misión. Otros optaron por conformarse con una distancia de despegue poco competitiva. En cualquier caso, estas decisiones implicaron una pérdida de eficiencia y/o de prestaciones en la aeronave, pero al fin y al cabo permitían que la aeronave operara, aunque eso sí a un mayor coste.

Por el contrario, de las tres limitaciones mencionadas, la relacionada con la estabilidad de la aeronave fue la que más esfuerzos concentró durante las primeras fases de desarrollo de esta tecnología, ya que una aeronave no estable era simplemente imposible de operar e impedía que el concepto fuese avanzando en otros ámbitos. Así pues, se probaron distintas configuraciones alares con el fin de mejorar las características de estabilidad, destacando la desarrollada por el ingeniero aerodinámico alemán Lippisch en la década de los 60.


\subsubsection{Características}
\label{sec:wig:lippisch:characteristics}

La configuración Lippisch fue una de las primeras en ofrecer buenas características de estabilidad en el sector de los ekranoplanos. El primer modelo en esta categoría fue el Lippisch X-112,
\rmfig{lippisch}{jpg}{120}{ht}{Lippisch X-112}
 . Sus principales características son:
\begin{itemize}
\item Ala delta invertida.
\item Alargamiento alar moderado.
\item Ángulo de dihedro negativo.
\item Cola en T situada en una posición elevada y de dimensiones relativamente reducidas.
\end{itemize}


Las principales ventajas de esta configuración son:
\begin{itemize}
\item Estabilidad en un amplio rango de combinaciones de ángulos de ataque y separaciones entre la aeronave y la superficie subyacente.
\item Elevada eficiencia aerodinámica, con una relación sustentación–resistencia del orden de 25. Un valor habitual en la aviación comercial actual se sitúa en torno a 17.
\item Posibilidad de realizar “saltos dinámicos”, es decir, abandonar temporalmente el área de influencia del efecto suelo para superar un obstáculo.
\end{itemize}

Sin embargo, los ekranoplanos de tipo Lippisch no son capaces, en general, de incorporar la tecnología PAR eficientemente, por lo que se ven obligados a usar motores sobredimensionados o bien cuentan con unas distancias de despegue poco competitivas.

