\subsection{Vehículos de referencia}

\mfig{lippischX114}{jpg}{135}{ht!}{Lippisch X-114}

\mfig{airfish3}{jpg}{135}{ht!}{Airfish 3}

\mfig{airfish8}{jpg}{135}{ht!}{Airfish 8}

\mfig{aquaglide5}{jpg}{135}{ht!}{Aquaglide-5}

\mfig{planosX114}{jpg}{140}{bt}{Vistas del Lippisch X-114}

\begin{table}[ht]
\centering
\caption{Dimensiones detallas del Lippisch X-114.}
\label{tab:x114dim}
\begin{tabular}{lr}
\toprule
Parámetro                                   & Valor       \\ \midrule
Cuerda media aerodinámica [m]               & $3.60$      \\
Cuerda en la raíz [m]                       & $5.30$      \\
Cuerda en la punta [m]                      & $0.70$      \\
Estrechamiento [–]                          & $0.13$      \\
Envergadura [m]                             & $7.00$      \\
Superficie alar [m$^2$]                     & $17.50$     \\
Alargamiento alar [–]                       & $2.80$      \\ \midrule
Cuerda m.a. cola horizontal [m]             & $1.10$      \\
Envergadura cola horizontal [m]             & $3.80$      \\
Superficie cola horizontal [m$^2$]          & $4.20$      \\ \midrule
Cuerda m.a. cola vertical [m]               & $1.20$      \\
Envergadura cola vertical [m]               & $1.20$      \\
Superficie cola vertical relativa [m$^2$]   & $1.58$      \\ \bottomrule
\end{tabular}
\end{table}

\FloatBarrier

\subsection{Piezas del modelo a escala}

\mfig{fuselage}{pdf}{100}{ht}{Vistas y cotas (en mm) del fuselaje}

\mfig{wing}{pdf}{140}{ht}{Vistas y cotas (en mm) del ala}

\mfig{vtail}{pdf}{100}{ht}{Vistas y cotas (en mm) de la cola vertical}

\mfig{htail}{pdf}{100}{ht}{Vistas y cotas (en mm) de la cola horizontal}

\FloatBarrier

\subsection{Otros}

\mfig{crt150}{jpg}{140}{ht}{Cortador CRT150 programable mediante control numérico}

\mfig{cintaelastica}{jpg}{140}{ht}{Cinta elástica perforada utilizada para el lanzamiento del modelo}

\mfig{cintapreparada}{jpg}{140}{ht}{Detalle de la cinta elástica con sus dos extremos acondicionados para el lanzamiento del modelo}

\mfig{alcayataarandela}{jpg}{140}{ht}{Detalle de la arandela y la cola utilizadas para mantener unida la alcayata al fuselaje y distribuir los esfuerzos}
