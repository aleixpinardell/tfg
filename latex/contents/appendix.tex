\mfig{lippischX114}{jpg}{140}{ht!}{Lippisch X-114}

\mfig{airfish3}{jpg}{140}{ht!}{Airfish 3}

\mfig{airfish8}{jpg}{140}{ht!}{Airfish 8}

\mfig{aquaglide5}{jpg}{140}{ht!}{Aquaglide-5}

\mfig{planosX114}{jpg}{140}{bt}{Vistas del Lippisch X-114}

\begin{table}[ht]
\centering
\caption{Dimensiones detallas del Lippisch X-114.}
\label{tab:x114dim}
\begin{tabular}{lr}
\toprule
Parámetro                                   & Valor       \\ \midrule
Cuerda media aerodinámica [m]               & $3.60$      \\
Cuerda en la raíz [m]                       & $5.30$      \\
Cuerda en la punta [m]                      & $0.70$      \\
Estrechamiento [–]                          & $0.13$      \\
Envergadura [m]                             & $7.00$      \\
Superficie alar [m$^2$]                     & $17.50$     \\
Alargamiento alar [–]                       & $2.80$      \\ \midrule
Cuerda m.a. cola horizontal [m]             & $1.10$      \\
Envergadura cola horizontal [m]             & $3.80$      \\
Superficie cola horizontal [m$^2$]          & $4.20$      \\ \midrule
Cuerda m.a. cola vertical [m]               & $1.20$      \\
Envergadura cola vertical [m]               & $1.20$      \\
Superficie cola vertical relativa [m$^2$]   & $1.58$      \\ \bottomrule
\end{tabular}
\end{table}

\mfig{fuselage}{pdf}{100}{ht}{Vistas y cotas (en mm) del fuselaje}

\mfig{wing}{pdf}{140}{ht}{Vistas y cotas (en mm) del ala}

\mfig{vtail}{pdf}{100}{ht}{Vistas y cotas (en mm) de la cola vertical}

\mfig{htail}{pdf}{100}{ht}{Vistas y cotas (en mm) de la cola horizontal}

\mfig{crt150}{jpg}{140}{ht}{Cortador CRT150 programable mediante control numérico}

\mfig{cortetemp}{jpg}{140}{ht}{Detalle de un corte en el que la temperatura del hilo es excesiva}

\mfig{assembledmodel}{jpg}{140}{ht}{Modelo ensamblado}

\mfig{cintaelastica}{jpg}{140}{ht}{Cinta elástica perforada utilizada para el lanzamiento del modelo}

\mfig{cintapreparada}{jpg}{140}{ht}{Detalle de la cinta elástica con sus dos extremos acondicionados para el lanzamiento del modelo}

\mfig{alcayataarandela}{jpg}{140}{ht}{Detalle de la arandela y la cola utilizadas para mantener unida la alcayata al fuselaje y distribuir los esfuerzos}

\mfig{ajustek}{pdf}{90}{ht}{Obtenció de la constante elástica mediate ajuste de los datos obtenidos por una recta}

\mfig{lanzamientonoge}{jpg}{130}{ht}{Estudio del comportamiento del modelo fuera de la zona de influencia del efecto suelo}

\clearpage

\begin{table}[ht]
\centering
\caption{Distancia recorrida por el modelo bajo la influencia del efecto suelo en función de la elongación de la cinta elástica. Lanzamientos en sentido norte.}
\label{tab:test1geN}
\begin{tabular}{lll}
\toprule
\# Prueba      & Elongación [cm] & Distancia [m]    \\ \midrule
1Na              & $30$           & $4.5$             \\
1Nb              & $30$           & $5.2$             \\
1Nc              & $30$           & $5.7$             \\ \hline
2Na              & $40$           & $7.8$                \\
2Nb              & $40$           & $7.3$                \\
2Nc              & $40$           & $7.3$                \\ \hline
3Na              & $50$           & $11.3$                \\
3Nb              & $50$           & $11.0$                \\
3Nc              & $50$           & $11.0$                \\ \bottomrule
\end{tabular}
\end{table}

\begin{table}[ht]
\centering
\caption{Distancia recorrida por el modelo bajo la influencia del efecto suelo en función de la elongación de la cinta elástica. Lanzamientos en sentido sur.}
\label{tab:test1geS}
\begin{tabular}{lll}
\toprule
\# Prueba      & Elongación [cm] & Distancia [m]    \\ \midrule
1Sa              & $30$           & $7.5$             \\
1Sb              & $30$           & $7.5$             \\
1Sc              & $30$           & $7.0$             \\ \hline
2Sa              & $40$           & $9.6$                \\
2Sb              & $40$           & $9.7$                \\
2Sc              & $40$           & $10.0$                \\ \hline
3Sa              & $50$           & $11.1$                \\
3Sb              & $50$           & $11.9$                \\
3Sc              & $50$           & $11.4$                \\ \bottomrule
\end{tabular}
\end{table}

\begin{table}[ht]
\centering
\caption{Valores medios de la distancia recorrida por el modelo bajo la influencia del efecto suelo en función de la elongación de la cinta elástica.}
\label{tab:test1ge}
\begin{tabular}{lll}
\toprule
\# Prueba      & Elongación [cm] & Distancia [m]    \\ \midrule
1N              & $30$           & $5.1$             \\
2N              & $40$           & $7.5$             \\
3N              & $50$           & $11.1$             \\ \hline
1S              & $30$           & $7.3$                \\
2S              & $40$           & $9.4$                \\
3S              & $50$           & $11.5$                \\ \hline
1              & $30$           & $6.2$                \\
2              & $40$           & $8.5$                \\
3              & $50$           & $11.3$                \\ \bottomrule
\end{tabular}
\end{table}

\mfig{grabacionarriba}{jpg}{140}{ht}{Modelo a punto de ser lanzado sobre una superficie en la que se ha marcado una escala}

\mfig{d5}{jpg}{140}{ht}{Cuatro fotogramas consecutivos de una de las pruebas realizadas para determinar la velocidad del modelo a 5 metros del origen}

\begin{table}[ht]
\centering
\caption{Distancia recorrida por el modelo fuera de la zona de influencia del efecto suelo en función de la elongación de la cinta elástica.}
\label{tab:testnoge}
\begin{tabular}{lll}
\toprule
\# Prueba      & Elongación [cm] & Distancia [m]    \\ \midrule
4a              & $30$           & $2.3$             \\
4b              & $30$           & $1.9$             \\ \bottomrule
\end{tabular}
\end{table}

\mfig{noge20cm}{jpg}{140}{ht}{Configuración experimental para las pruebas fuera de la zona de influencia del efecto suelo}

\mfig{nogegeom}{jpg}{140}{ht}{Construcción geométrica para la determinación de la distancia a la que el modelo regresa a su altura de vuelo inicial}

\mfig{groundclearance}{jpg}{140}{ht}{Primer plano del modelo durante una prueba en el que se observa su separación vertical al suelo}


