El uso de procedimientos analíticos para el cálculo de las prestaciones de una aeronave puede servir a modo de estimación, pero en cualquier caso deben ser complementadas con pruebas experimentales en las que realmente se evalúan las prestaciones reales del vehículo.

Además, estas técnicas suelen ser muy complejas matemáticamente y a menudo es necesario realizar simplificaciones y suposiciones que acaban reduciendo la fiabilidad y precisión de los cálculos. En el caso de un vehículo que opera bajo la influencia del efecto suelo, la complejidad y las incertidumbre se acentúan todavía más.

Por ello, con el objetivo de poder estudiar experimentalmente las prestaciones de un ekranoplano de tipo Lippisch, se ha procedido a construir un modelo a escala. En la siguiente sección de detallarán dichas pruebas y se expondrán los resultados obtenidos. Pero antes, en esta sección, se va a explicar qué técnicas y materiales de construcción se han utilizado, justificando su elección.


\subsection{Concepción del modelo}
\label{sec:building:conception}

Antes de elegir el material principal a utilizar en el modelo, es imprescindible definir su misión, es decir, conviene planear las pruebas a las que va a ser sometido con el fin de no construir un modelo que sea incompatible con las mismas. Por ejemplo, si el modelo va a ser probado sobre la superficie del agua, quedaría descartado por completo el uso de materiales como la cartulina o la madera, ya que al ser absorbentes la forma y/o el peso del modelo se podrían ver modificados al entrar en contacto con el agua.

\subsubsection{Planta propulsora}
\label{sec:building:conception:propulsion}

Asimismo, se debe decidir si el modelo será autopropulsado o no. Esta decisión afectará al modo en que se diseñan las pruebas. La decisión adoptada ha sido la de construir un modelo no propulsado, con el fin de simplificar el proceso de construcción y de abaratar costes. Más adelante se propondrá la adición de uno o varios motores en el modelo, el cual podría volver a ser utilizado en trabajos futuros.

Ante la decisión de no incluir una planta propulsora, surge la necesidad de construir el modelo a escala de tal forma que pueda ser “lanzado” con una cierta velocidad inicial conocida. Para ello, se puede hacer uso de cintas o gomas elásticas calibradas, como se detallará en la siguiente sección. Esto implica que el modelo debe poder ser enganchado a la cinta elástica mientras está esté tensa y debe liberarse automáticamente en cuanto la cinta regrese a una posición no excitada.

\subsubsection{Material}
\label{sec:building:conception:material}

La fuerza elástica de la cinta se aplicará de forma puntual sobre el modelo. Ya sea mediante el diseño de un mecanismo que lance el modelo como un tirachinas lanza o una piedra, o mediante la adición de un gancho en la parte frontal de modelo, la tensión aplicada será elevada y por lo tanto será necesario el uso de un material medianamente resistente y difícilmente deformable.

El poliestireno expandido (\rmfig{eps}{jpg}{90}{ht}{Poliestireno expandido}) es un material que se utiliza mucho en aeromodelismo para realizar pruebas en una fase inicial del diseño. Su bajo coste y densidad, su resistencia a impactos y su bajo nivel de absorción lo convierten en un buen candidato para el modelo a construir.

Además, la elección de esta material tan ligero (su densidad se sitúa entre los 10 y 25 g/m$^3$)\cite{ref:poliestirenoexpandido}, en comparación a otros materiales como maderas, plásticos o metales, permite la construcción de un modelo sin cavidades en su interior, es decir, tanto las superficies alares como el fuselaje pueden construirse a partir de un bloque compacto de densidad uniforme, simplificando notoriamente el proceso de construcción. Si se utilizase otro material más pesado, las piezas deberían ser huecas para no exceder el peso de diseño y deberían ser reforzadas internamente.

En cuanto al ensamblaje de las piezas fabricadas a partir de poliestireno expandido, existen colas que permiten pegar este material fácilmente, simplificando el proceso y evitando el uso de remaches, soldaduras u otros mecanismos de unión que añadirían peso al modelo y lo someterían a esfuerzos puntuales elevados.


\subsection{Diseño de las partes}
\label{sec:building:parts}

Una aeronave real se compone de cientos o miles de piezas fabricadas independientemente que posteriormente se ensamblan para dar lugar al producto acabado. Sin embargo, para la construcción del modelo a escala simplificado, se considerarán únicamente cuatro partes: el fuselaje, el ala, la cola vertical y la cola horizontal.

En las subsecciones sucesivas se detallan las características de cada una de estas cuatro partes que conforman la totalidad del modelo. En la \rmfig{model}{pdf}{90}{ht}{Modelo a escala simplificado} se pueden identificar cada una de las partes en el modelo.

\subsubsection{Fuselaje}
\label{sec:building:parts:fuselage}
\mfig{fuselage}{pdf}{90}{ht}{Vistas y cotas (en mm) del fuselaje}
\FloatBarrier

\subsubsection{Ala}
\label{sec:building:parts:wing}
\mfig{wing}{pdf}{140}{ht}{Vistas y cotas (en mm) del ala}
\FloatBarrier

\subsubsection{Cola vertical}
\label{sec:building:parts:vtail}
\mfig{vtail}{pdf}{90}{ht}{Vistas y cotas (en mm) de la cola vertical}
\FloatBarrier

\subsubsection{Cola horizontal}
\label{sec:building:parts:htail}
\mfig{htail}{pdf}{90}{ht}{Vistas y cotas (en mm) de la cola horizontal}
\FloatBarrier


\subsection{Corte de las partes}
\label{sec:building:technique}

En un primer lugar, se intentó la construcción de las partes mediante corte manual. En el caso del poliestireno expandido, los principales inconvenientes fueron los ya mencionados: la dificultad para lograr una buena precisión y sobre todo un acabado rugoso (la presencia de las famosas burbujas que se desprenden fácilmente). Por ello, se probó otro material, el poliestireno extruido, ligeramente más denso\cite{ref:poliestirenoextruido}, pero sin el inconveniente de las “burbujas”. Sin embargo, esta opción también se tuvo que descartar por la imposibilidad de alcanzar niveles de precisión aceptables.

Tras haber descartado la técnica de corte manual en frío, se identificó una posible solución: el corte mediante hilo caliente. La Escuela Técnica Superior de Ingeniería del Diseño dispone de un taller que cuenta con un cortador CRT150 de Alarsis (\rmfig{crt150}{jpg}{120}{ht}{Cortador CRT150 programable mediante control numérico}) programable mediante control numérico. Según el fabricante, la máquina tiene una resolución de $0.02$ mm\cite{ref:crt150}, aunque en la práctica la precisión varía en función de algunos parámetros como la temperatura y velocidad de corte o la forma de la pieza.

\subsubsection{Funcionamiento}
\label{sec:building:technique:functioning}

El principio de funcionamiento de la máquina de corte es muy sencillo: en primer lugar, se sitúa un bloque de poliestireno sobre la mesa de corte (formada por las dos tablas que se pueden observar en la \rfig{crt150}) y se sitúa el hilo en la posición deseada. El hilo se extiende de izquierda a derecha tal y como se observa en la fotografía. La máquina cuenta con unas cintas que permiten mover el hilo en el plano de la mesa de corte, en la dirección longitudinal de las tablas y tanto en sentido entrante como saliente. Además, el hilo posee un segundo grado de libertad, ya que puede desplazarse verticalmente.

Cada uno de los extremos del hilo puede controlarse independientemente del otro, lo que le otorga una gran versatilidad a la máquina. Por ejemplo, uno de los extremos puede permanecer en el fondo de la mesa de corte mientras el otro se desplaza hasta el frente de la mesa de corte, dando lugar a un corte oblicuo.

Tanto el desplazamiento del hilo como su velocidad se controlan desde el ordenador que se muestra en la parte derecha de la fotografía. Mediante el software CeNeCe Pro 1.12, es posible indicar las distancias a recorrer por cada uno de los cabezales, así como la velocidad de desplazamiento. Basándose en experiencia previa, el técnico del taller propone realizar movimientos de posicionamiento (que no implican corte de material) a la velocidad máxima permitida, 30 mm/s, mientras que para el corte recomienda una velocidad de 8 mm/s.

El software permite, además del control de la posición del hilo mediante la introducción de distancias a través del teclado, su control automático mediante la importación de archivos generados por sotfware CAD. Esto permite programar la máquina para que el hilo siga los contornos deseados tras haberlo situado en la posición de inicio de corte adecuada y haber posicionado debidamente el bloque de poliestireno precortado. Esta funcionalidad y sus limitaciones se detallarán más adelante.

Además de la distancia y la velocidad, otro parámetro importante es la temperatura de corte. Ésta se puede regular mediante una ruleta analógica situada al lado de la torre del ordenador. En operaciones sencillas, no tiene excesiva influencia y existe un rango amplio de valores para el cual el acabado obtenido es bueno. Sin embargo, para el corte de piezas más complejas, puede resultar complejo determinar la temperatura ideal, ya que posiblemente ésta sea distinta para distintas partes de las piezas. Para las piezas necesarias para el modelo la ruleta se ha situado habitualmente entre las posiciones 3 y 5, en una escala del 0 al 10, donde 10 es la temperatura máxima.

En la mayoría de los casos resulta imprescindible hacer uso de pesos, que se sitúan encima del bloque de material con el fin de que el hilo no lo desplace de su posición original, tal y como se observa en la \rmfig{cortepeso}{jpg}{120}{ht}{Bloque de poliestireno sujetado por un peso metálico durante una opreación de corte}.


\subsubsection{Limitaciones}
\label{sec:building:technique:limitations}

Mediante el cortador de hilo caliente se solucionan dos de los problemas mencionados anteriormente: es posible alcanzar una precisión aceptable y se consigue un buen acabado, libre de “burbujas”. Sin embargo, cuenta con algunas limitaciones, asociadas fundamentalmente al tamaño y a la forma de la pieza que se pretende cortar.

La limitación dimensional no debería tener consecuencias sobre la construcción de las partes, ya que al tratarse de un modelo a escala, la más grande de las piezas, el fuselaje (con una longitud aproximada de 52 cm), no superaría las dimensiones máximas permitidas, ya que el hilo caliente se puede desplazar en un espacio tridimensional de dimensiones 83x80x37 cm, según información proporcionada por el técnico del taller. La menor de las distancias se refiere a la dimensión vertical.

Sin embargo, sí existe una limitación importante a la hora de programar el corte a ser realizado por la máquina. No es posible cortar todo tipo de formas, sino que más bien el espectro de posibilidades viene limitado principalmente por la existencia de tres tipos de cortes:
\begin{itemize}
\item \textbf{Corte simple}. El hilo se sitúa en la posición deseada (mediante el uso de un programa informático) y se le indica la distancia que debe desplazarse, así como la dirección en la que debe hacerlo. Este corte suele ser útil en la preparación de bloques de poliestireno con las dimensiones adecuadas para efectuar posteriormente cortes más complejos.
\item \textbf{Corte por extrusión}. Se le proporciona un perfil al programa informático y el hilo sigue el contorno marcado por este perfil. La característica principal de esta operación de corte es que la pieza resultante dispone de sección constante en el plano perpendicular al vector director del hilo.
\item \textbf{Corte por lofting}. En este caso, se le proporciona al software dos perfiles distintos, y el hilo recorre en cada cara del bloque de poliestireno cada uno de los contornos de forma síncrona. Sin embargo, esta opción cuenta con una limitación importante: si uno de los perfiles dispone de un perímetro mucho menor que el otro, el hilo permanecerá en dicha cara demasiado tiempo en posiciones cercanas, y debido a su alta temperatura se “comerá” excesivo material (\rmfig{cortetemp}{jpg}{120}{ht}{Detalle de un corte en el que la temperatura del hilo es excesiva}). Bajando la temperatura de corte para evitar dicho problema en el contorno de menor perímetro, el problema aparecerá en la cara opuesta, donde una temperatura insuficiente impedirá el correcto avance del hilo a través del material.
\end{itemize}


\subsubsection{Adaptación de las piezas}
\label{sec:building:technique:adaptation}

Debido a las limitaciones mencionadas en la subsección precedente, algunas de las piezas han debido de ser adaptadas o incluso divididas en distintas partes para poder ser cortadas mediante la técnica de hilo caliente. A continuación se exponen los detalles de las adaptaciones llevadas a cabo.

\paragraph{Fuselaje.} El fuselaje se puede obtener como una extrusión de un perfil suave y continuo, por lo que no ha sido necesario realizar ninguna adaptación en esta pieza.

\paragraph{Ala.} El ala es la pieza más compleja de todo el modelo. El primer paso pasa por dividirla en semiala derecha y semiala izquierda. Introduciendo los datos relativos a una de las dos semialas, la otra se puede obtener fácilmente por simetría. Sin embargo, todavía no es posible la construcción de una semiala en una única operación de corte. Esta pieza no se ajusta ni a una extrusión ni a un lofting, ya que estas operaciones trabajan a partir de contornos situados en planos paralelos. Por ello, se ha tenido que devidir la semiala en dos partes: la parte anterior y la parte posterior (\rmfig{wingfrontback}{pdf}{90}{ht}{Partes anterior y posterior de la semiala izquierda}).

\paragraph{Parte anterior de la semiala.} Esta pieza se puede obtener mediante un lofting de dos contornos con perimétros similares situados en caras paralelas, por lo que a priori no sería necesario realizar más adaptaciones. Sin embargo, durante el corte surgieron algunos problemas, relacionados con el hecho de que para definir uno de los contornos sea necesarios una cantidad de puntos superior que para el otro. Esto ocasiona problemas de sincronismo en el corte, que finalmente se solucionaron simplificando el borde de ataca del ala, eliminando su redondeo, de modo que tanto el perfil en la raíz como en la punta del ala se podían definir mediante el mismo número de puntos: cuatro.

\paragraph{Parte posterior de la semiala.} Esta pieza es una de las más complejas de todo el modelo. Para poder empezar a abordarla, es necesario identificar dos planos paralelos que permitan definir una operación de lofting. Los planos elegidos son el plano en el que se une la parte anterior de la semiala a la parte posterior y el plano paralelo a éste que contiene el borde fuga de la semiala, que consiste en un único punto. Sin embargo, como ya se ha comentado, este corte generaría una pieza “quemada” cerca del borde de fuga, ya que el hilo se mantendría estacionario en el punto correspondiente al borde de fuga mientras recorre el contorno en el plano opuesto. Para solucionar este problema, se ha procedido a extender la sección de la punta de ala (\rmfig{wingconstruction}{png}{110}{ht}{Extensión de la sección de punta de ala}) hasta que el contorno obtenido en el plano posterior dispusiera de un perímetro lo suficientemente grande como para que la pieza resultante no resultase “quemada”. A continuación, una vez obtenida la pieza “extendida”, mediante una operación de corte sencilla, se ha eliminado la parte sobrante, obteniendo finalmente la pieza deseada.

\paragraph{Cola vertical.} Esta pieza también ha generado dificultades durante su construcción. Es evidente que se trata de una pieza generable por lofting, situándose uno de los planos en la parte inferior del fuselaje y el otro, paralelo a éste, a la altura de la cola horizontal. En este caso, no existe el riesgo de “quemar” la pieza ya que las dos secciones cuentan con perímetros del mismo orden de magnitud. Sin embargo, ha vuelto a surgir el problema de sincronismo mencionado en el parágrafo correspondiente a la construcción de la parte anterior de la semiala: al tratarse de secciones distintas, cada una viene definida por una cantidad distinta de puntos, lo cual hace que el software envíe ordenes de corte inconsistentes a la máquina. Por ello, se ha decidido modificar el perfil superior, sustituyéndolo por una copia a escala del perfil inferior, una función integrada en el software que no genera problemas de sincronismo. A continuación, se ha recortado el contorno del fuselaje sobre la parte inferior de la pieza obtenida, con el fin de eliminar el material de la cola vertical que quedaría incrustado en el fuselaje.

A pesar de haber realizado dichas adaptaciones, la construcción de esta pieza ha seguido dando problemas. La mesa de trabajo tiene unas dimensiones de 83x80 cm, es decir, el ángulo máximo que se puede inclinar el hilo es:
\eq{maxangle}{
\alpha_{\max} = \arctan{\frac{83}{80}} \cong 46$ deg$
}
La cola vertical, como se incida en la \rfig{vtail}, presenta un valor del ángulo de flecha (máximo en el borde de ataque) de 46 grados, por lo que debería posicionarse el bloque con infinita precisión para que ninguno de los cabezales llegase al final de su recorrido antes de haber completado el corte. Tras varios intentos, se ha descartado esta opción.

Para superar este problema, se han proyectado los perfiles inferior y superior sobre planos perpendiculares al borde de fuga, que tiene una flecha de $28.2$ grados. Con ello se ha conseguido que la inclinación del hilo durante su recorrido por el borde de fuga sea de $46 - 28.2 = 17.8$ grados en lugar de 46. Una vez obtenida la pieza, se han eliminado el material sobrante (las partes proyectadas) mediante dos cortes sencillos, obteniendo finalmente la pieza deseada.

\paragraph{Cola horizontal.} Como el fuselaje, esta pieza se puede obtener mediante una sencilla extrusión, por lo que no ha requerido ninguna adaptación.


\subsection{Ensamblaje}
\label{sec:building:assembly}

Una vez superadas todas las limitaciones encontradas durante el proceso de corte de las distintas piezas que conforman el modelo a escala (\rmfig{allparts}{jpg}{100}{ht}{Partes del modelo a escala antes de ser ensambladas. De izquierda a derecha y de arriba a abajo: cola horizontal, cola vertical, semiala derecha, fuselaje y semiala izquierda}), se puede proceder al ensamblaje del modelo completo. Por recomendación del técnico del taller, se ha usado cola Pattex 100\% para unir las piezas entre sí.

El procedimiento seguido ha sido el siguiente:
\begin{enumerate}
\item Unión de las parte anterior y posterior de cada una de las semialas.
\item Unión de cada una de las semialas al fuselaje.
\item Unión de la cola horizontal a la cola vertical.
\item Unión de la cola al fuselaje.
\end{enumerate}

Tras la aplicación de cola entre las superficies de dos piezas distintas, se ha requerido una espera de alrededor de 30 minutos para una correcta unión. Durante este tiempo, se ha presionado una pieza contra la otra para mejorar el resultado final del ensamblaje. El modelo ensamblado se muestra en la \rmfig{assembledmodel}{jpg}{120}{ht}{Modelo ensamblado}
\FloatBarrier
