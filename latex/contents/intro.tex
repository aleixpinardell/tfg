Se denomina ekranoplano a un vehículo diseñado para utilizar de forma continuada la mejora de eficiencia aerodinámica conseguida al volar cerca de una superficie. Estos vehículos se denominan en inglés \emph{wing in ground (WIG) effect vehicles} y, aunque se empezó a experimentar con este concepto ya en la década de los años 30\cite{ref:kaario}, no fue hasta los años 60 cuando consiguieron cierta relevancia gracias a los diseños realizados en la antigua Unión Soviética por ingenieros como Rostislav Alexeev\cite{ref:alexeev}, dando lugar a uno de los ekranoplanos más icónicos por sus grandes dimensiones y capacidad de carga, conocido con el nombre de Monstruo del Mar Caspio (\rfig{kmseamonster}), ya que era en este lugar donde se llevaban a cabo las pruebas experimentales.

\mfig{kmseamonster}{jpg}{110}{h}{El Monstruo del Mar Caspio, el primer ekranoplano ruso de grandes dimensiones}

El principio de funcionamiento de estos vehículos se basa en que cuando un cuerpo aerodinámico vuela cerca de una superficie experimenta una mejora de su eficiencia aerodinámica, efecto que es conocido con el nombre de efecto suelo. Si bien este fenómeno se utiliza ocasionalmente en algunas fases de vuelo de corta duración, su uso continuado durante la mayor parte del vuelo nunca ha llegado a popularizarse.

Posiblemente, la principal limitación de los ekranoplanos sea la necesidad de volar a unos pocos metros o centímetros de una superficie subyacente, restringiendo en la práctica su uso a vuelos cerca del agua, los cuales se llevan a cabo a velocidades sensiblemente inferiores a las de la aviación comercial.

Probablemente por dicha razón nunca hayan logrado destacarse sobre las aeronaves convencionales. Sin embargo, el potencial de los ekranoplanos no se halla en desplazar o sustituir la aviación comercial, actualmente muy competitiva, sino en revolucionar el transporte marítimo al ser capaz de completar, en un tiempo mucho menor y de una manera mucho más eficiente, algunos de los trayectos que hoy en día realizan las embarcaciones.

La razón de ser de los ekranoplanos es cubrir un espacio del mercado del transporte intercontinental situado a medio camino entre las aeronaves convencionales y las embarcaciones marítimas. Si en el futuro se consigue diseñar un vehículo seguro que sea más veloz y más barato que una embarcación, es muy probable que pueda revolucionar el transporte marítimo del mismo modo que la aviación comercial revolucionó el transporte de pasajeros.

Ante este prometedor potencial, con este trabajo se busca establecer las bases teóricas y prácticas necesarias para el diseño y construcción de un vehículo de estas características. Tras un breve repaso histórico, en el que se prestará especial atención al trabajo realizado por el ingeniero aerodinámico Alexander Lippisch, se abordará el diseño conceptual de un ekranoplano, adaptando las técnicas de diseño propuestas por Raymer\cite{ref:raymer} y Roskam\cite{ref:roskam} a esta nueva tipología de vehículo.

Una vez definidas las principales características del ekranoplano, se abordará la construcción de un modelo a escala del mismo. El material utilizado será el poliestireno expandido, y la técnica de construcción que se empleará será la de corte por hilo caliente\cite{ref:hotwirecut}. Se espera que de la información publicada en este trabajo se puedan beneficiar futuros graduandos que decidan aventurarse también a la construcción de modelos a escala de sus diseños.

Una vez construido el modelo a escala, se realizarán una serie de pruebas sobre el mismo para corroborar su mejora de prestaciones cuando vuela bajo la influencia del efecto suelo y se estudiará su estabilidad dinámica para distintas configuraciones de peso, con el objetivo de demostrar que este tipo de vehículo puede ser seguro y eficiente, y estableciendo al mismo tiempo un punto de partida para trabajos futuros que quieran mejorar el concepto aquí propuesto o realizar pruebas adicionales sobre el mismo.
