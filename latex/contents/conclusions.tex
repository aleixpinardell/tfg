\subsection{Conclusiones}
\label{sec:conclusions:conclusions}

Al inicio de este trabajo, se establecía como objetivo del mismo la determinación experimental de las prestaciones en eficiencia y estabilidad de un ekranoplano de tipo Lippisch. Se ha podido observar, mediante la realización de pruebas sobre un modelo a escala, los siguientes comportamientos:

\begin{itemize}
\item Con un mismo impulso inicial, el modelo es capaz de mantener su altura de vuelo durante un período de tiempo mucho mayor si vuela en la zona de influencia del efecto suelo. Si lo hace fuera de ella, la distancia que recorre antes de recuperar la altitud a la que ha sido lazando se ve reducida a menos de la mitad. Esta mejora de eficiencia, trasladada a un vehículo real autopropulsado, se traduce en un alcance potencialmente superior cuando se vuela cerca de la superficie marítima.
\item El modelo es longitudinalmente estable para una gran variedad de velocidades y configuraciones de peso, aun habiendo empleado técnicas de construcción económicas que no destacan por gozar de una gran precisión y habiendo realizado simplificaciones tanto durante el proceso de diseño como de construcción. Se puede esperar, por tanto, que un vehículo WIG diseñado y construido cuidadosamente en base a la configuración propuesta por A. Lippisch goce de buenas prestaciones de estabilidad.
\item La separación vertical existente entre la parte inferior del modelo y la superficie subyacente durante la mayor parte del vuelo es del orden del 11\% de la cuerda media aerodinámica. Se trata de una altitud de equilibrio, por encima de la cual el efecto suelo disminuye, haciendo caer la sustentación y al vehículo, y por debajo de la cual la acentuación del efecto suelo comporta incrementos en la sustentación que hacen recuperar al vehículo esta separación de equilibrio.
\item Las posibles configuraciones del vehículo según la distribución de la carga de pago y el combustible es un factor importante a tener en cuenta durante la fase de diseño, pues se ha observado que, aunque el modelo goza de total estabilidad en condiciones de peso operativo en vacío, ciertas configuraciones de carga han provocado que su estabilidad empeore y aparezcan problemas como el \emph{pitch-up}.
\end{itemize}

Antes de la realización de las pruebas sobre el modelo, ha sido necesario llevar a cabo un proceso de construcción, del cual se pueden extraer las siguientes conclusiones, que pueden resultar útiles para cualquiera que se proponga construir un modelo mediante las mismas técnicas expuestas en este trabajo:
\begin{itemize}
\item El poliestireno expandido es un material con una buena relación resistencia/peso, ideal para ser utilizado en la construcción de modelos que no se espera que vayan a sufrir impactos importantes durante su vida útil.
\item Para el corte de poliestireno expandido por hilo caliente es necesario disponer previamente de todas las piezas que se van a necesitar recortar como una extrusión de un perfil o un \emph{lofting} entre dos perfiles situados en caras paralelas entre sí. Si alguna de las piezas no es generable mediante estas operaciones, se deberá intentar dividir la pieza en varias partes que sí lo sean, o se deberán encadenar operaciones cuya superposición dé lugar a la pieza deseada.
\end{itemize}


\subsection{Trabajos futuros}
\label{sec:conclusions:future}

Como ya se ha mencionado anteriormente, este trabajo no pretende ser un proyecto acabado en el que no haya lugar para futuras evoluciones o mejores, más bien todo lo contrario. Con este trabajado se han establecido las bases que en un futuro pueden servir para el desarrollo de un modelo de ekranoplano más perfeccionado que el aquí presentado. Algunas de las características a añadir al modelo en el futuro podrían ser:
\begin{itemize}
\item \textbf{Unidad de medida inercial}. Existen chips de pequeñas dimensiones y reducido peso que podrían ser incorporados al modelo para poder medir con mayor precisión algunos parámetros de relevancia, como su separación vertical con el suelo, las aceleraciones experimentadas, la actitud y la respuesta en estabilidad durante el vuelo.
\item \textbf{Dispositivos de punta alar}. La mayoría de los aviones comerciales, incorporan elementos aerodinámicos en las puntas del ala —\emph{winglets}— para reducir la resistencia inducida. En el caso de los ekranoplanos, se suelen utilizar dispositivos de mayores dimensiones, dando lugar a un incremento de la envergadura efectiva e incluso acentuando el efecto suelo bajo ciertas condiciones. En algunos estudios, como en \cite{ref:endplates}, se ha ido un paso más allá incorporando \emph{endplates} en la punta alar, lo cual dificulta el escape del aire a alta presión situado debajo del vehículo. En el futuro podría ser interesante estudiar el efecto que tiene sobre las prestaciones de un ekranoplano la incorporación de \emph{winglets} y/o \emph{endplates} en distintas configuraciones.
\item \textbf{Planta propulsora}. La inclusión de uno o varios motores sobre el modelo permitiría diseñar una configuración experimental totalmente distinta a la expuesta en este trabajo. El modelo podría ser testado a velocidad constante y, además, se mejoraría la validez de los resultados al aparecer una resistencia asociada a la presencia del motor, la cual también existirá en el vehículo real.
\end{itemize}

Independientemente de que dichos elementos se acaben añadiendo o no a posibles futuros modelos derivados del presentado en este trabajo, existen una serie de pruebas que podrían resultar de interés que no se han desarrollado en esta ocasión por estar fuera de los objetivos de este trabajo. Para el futuro se propone comprobar el comportamiento del modelo en los siguientes entornos:
\begin{itemize}
\item \textbf{Túnel de viento}. En este trabajo se ha renunciado a hacer uso de un túnel de viento porque la prueba principal, consistente en la determinación de la distancia recorrida para distintas configuraciones, se podía llevar a cabo mediante el diseño de una lanzadera. Sin embargo, si en el futuro se quiere profundizar en la medición de las prestaciones del modelo de ekranoplano fuera de la zona de influencia del efecto suelo, podría ser conveniente hacer uso de un túnel de viento, reduciendo el riesgo para la integridad estructural del modelo que conlleva el hecho de lanzarlo desde una cierta altura. Además, una configuración experimental en túnel de viento permitiría probar el modelo alejado de cualquier superficie y medir las fuerzas que actúan sobre él con mayor precisión, mediante el uso, por ejemplo, de galgas extensiométricas, y determinar si para ciertas velocidades aparecen fenómenos aeroelásticos peligrosos. Sin embargo, podría ser necesario reforzar el modelo o asegurarse, antes de introducirlo en la cámara de pruebas, de que va a ser capaz de resistir las tensiones a las que va a ser sometido durante las pruebas.
\item \textbf{Superficie líquida}. El potencial de los ekranoplanos se halla en ofrecer un transporte más competitivo que el de las embarcaciones marítimas, por lo que parece conveniente probarlo también sobre el agua, pues al fin y al cabo es el medio sobre el que va a operar el vehículo real. Aquí se podría observar si el comportamiento es similar al observado cerca de una superficie sólida, y sería posible estudiar también su respuesta a obstáculos en la superficie subyacente, principalmente olas. Del mismo modo que el modelo se ha escalado, las olas también deberían escalarse desde las dimensiones que se espera que tengan en el mar, prestando especial atención a algunos parámetros como la amplitud de la ola y a la distancia entre picos o valles.
\end{itemize}


\subsection{Conclusiones personales}
\label{sec:conclusions:personal}

Una de las experiencias más estimulantes y gratificantes a nivel personal durante el desarrollo de este Trabajo de Fin de Grado ha sido, sin duda, la construcción del modelo a escala. A pesar de tratarse un modelo bastante simplificado, no ha resultado para nada sencillo convertir los diseños existentes en el ordenador en un objeto tangible y funcional.

Las dificultades halladas durante la construcción del modelo radican en las limitaciones de la técnica utilizada y, especialmente, en el software encargado de gestionar los cortes por hilo caliente. La máquina utilizada se usa habitualmente con otros fines no aeronáuticos, en el ámbito de la construcción y el marketing, y raramente se usan para la construcción completa de un vehículo aéreo. De ahí que haya sido necesario efectuar varias simplificaciones y operaciones de corte complejas.

Sin embargo, a pesar de la dificultades halladas, ha sido posible obtener un modelo a escala funcional con el cual se han podido completar los principales objetivos marcados al inicio de este Trabajo de Fin de Grado. Tras la construcción de este modelo, estoy convencido de que si en el futuro debiera volver a utilizar la técnica del corte por hilo caliente, su uso me parecería mucho más natural e intuitivo.

Éste es pues el principal aprendizaje que obtengo del desarrollo de este trabajo, un conocimiento que tan solo se puede adquirir experimentando, cuando te ves forzado a ingeniártelas para superar las dificultades que se van presentando en el camino que has marcado. Un conocimiento práctico que considero un complemento fundamental a toda la sabiduría teórica adquirida a lo largo del desarrollo del Grado en Ingeniería Aeroespacial.

